\section{Methods} \label{sec:methods}

Choosing an algorithm for classifying data is difficult, because
different algorithms have different characteristics. There is no single
algorithm that is always best---each algorithm has its own strengths and
weaknesses. With this realization, we decided to use an ensemble method,
combining different algorithms.

We decided to choose three algorithms, so as to gain some redundancy
while keeping the method relatively simple. For any type of failure
\emph{intrinsic} to an algorithm, the two other algorithms are not prone
to make the same mistake. Thus most intrinsic mis-classifications can be
avoided; while one algorithm claims the wrong class, the two others get
it right.

The three algorithms we use all output continuous values from 0 to 1. Even in the
case that two of the algorithms are uncertain but leaning towards the
wrong classification, but one is correct and very certain, we can get
the correct result.

For combining the predictions of different algorithms, we used simply
the arithmetic mean. In our experiments, this proved to be efficient and
no other method showed significantly better results.

\subsection{Support Vector Machine}

The Support Vector Machine classifier works by finding a hyperplane that
separates two data samples with the largest possible margin. The method
becomes efficient as data is projected to a higher-dimensional space.
According to Cover's Theorem \cite{cover1965geometrical}, this makes finding a separating hyperplane
more probable. The actual mapping between the two spaces need not be
found--it suffices to define the inner product, i.e.~the kernel function
$\phi(u,v)$. \cite{cortes1995support}

Commonly, the kernel function is chosen to be linear $u' v$,
polynomial $(\gamma u' v + C)^n$, radial basis function
$\exp(-\gamma |u-v|^2)$ or sigmoid
$\tanh(\gamma u' v + C)$.

Weka \cite{weka} was used for applying the Support Vector Machine
classifier to the data. The specific implementation we used was LibSVM
\cite{libsvm}, using the WLSVM interface \cite{wlsvm}.

\subsection{Bernoulli Mixture}

If we know the likelihood functions $p(\mathbf{x}|\boldsymbol\theta_i)$
and prior probabilities $P(C_i)$ of classes $i$ we want to distinguish,
the Bayes' rule gives us an optimal classifier. The class of a new
instance $\mathbf{x}$ is given by choosing the class with the highest
posterior probability
\begin{equation}
 \textmd{arg}\max_i P(C_i|\mathbf{x}) \propto p(\mathbf{x}|\boldsymbol\theta_i)P(C_i).
\end{equation}

In reality, however, we do not have the true likelihood functions but
we need to estimate them from the data. This is what we do in this work
as well, we estimate likelihood functions for both ham and spam
instances and then use these models to classify new instances.

A natural way to model binary data and estimate its likelihood function
is to use a Bernoulli distribution, given by
\begin{equation}
  p(\mathbf{x}|\boldsymbol\mu) = \prod_{i=1}^D \mu_i^{x_i}(1-\mu_i)^{(1-x_i)}.
\end{equation}

Parameters $\mu_i$ are the means of different variables in data
$\mathbf{x}$ and thus the farther from the means we are the lower
likelihood we get.

A drawback of the Bernoulli distribution is, however, that it assumes
that the variables are independent. In the case of separating between spam
and ham, this assumption leads into in a Naive Bayes classifier. The
reason why this assumption is nevertheless commonly approved is that it
is infeasible to model all the covariances between variables as this
would require us to estimate a total of $\mathcal{O}(n^2)$ parameters.

To circumvent these problems, we use a
\emph{mixture of multivariate Bernoulli distributions} or shortly a
\emph{Bernoulli mixture} \cite{lazarsfeld1968latent}. A Bernoulli
mixture takes a linear combination of single Bernoullis, called the
mixture components, with parameters $\boldsymbol\mu_i$. The resulting
distribution is given by
\begin{equation}
 p(\mathbf{x}|\boldsymbol\mu,\boldsymbol\pi) = \sum_{k=1}^K \pi_kp(\mathbf{x}|\boldsymbol\mu_k),
\end{equation}

where $\pi_k$ are the weights that sum up to one. The covariance of this
distribution is not anymore a diagonal matrix meaning that it captures
some of the correlations in the data \cite{bishop2006pattern}.

To learn the model, we use the BernoulliMix software package
\cite{bmix}. BernoulliMix uses the EM algorithm to learn the model
parameters.

\subsection{Random Forest}

The Random Forest classifier, described by Ho \cite{ho1995random}, works
by creating a bunch of decision trees randomly. Each single tree is
created in a randomly selected subspace of the feature space. Trees in
different subspaces complement each other's classifications.

The single trees are Oblique Decision Trees. For each node of a tree, a
hyperplane is selected to divide the sample into two further
subspaces--corresponding to two branches of the tree. This is repeated
until each branch ends with a subspace containing only a single class.

Predictions of the different trees for a test point are combined with a
discriminant function, which in \cite{ho1995random} is basically done by
just averaging the posterior probabilities. The Random Forest classifier
attains an improved generalization accuracy compared to other decision
tree-based classifiers. It still retains the excellent accuracy on
training data inherent to decision trees. \cite{ho1995random}

The built-in Random Forest implementation in Weka \cite{weka} was used.
The implementation is created considering some further research on
generating Random forests \cite{breiman2001random}.
