\documentclass[12pt,a4paper,finnish,oneside]{book}
%\documentclass[12pt,a4paper,finnish,oneside,draft]{book} % luonnos, nopeampi
%\usepackage{tkk_t}
\usepackage[utf8]{inputenc}
\usepackage[T1]{fontenc}
\usepackage{alltt}
\usepackage[english,swedish,finnish]{babel}
\usepackage{amsmath}  % matematiikkaa..
\usepackage[pdfborder={0 0 0}]{hyperref}
\usepackage{color}
\usepackage{listings}
\usepackage{rotating}
\usepackage[format=hang,justification=raggedright]{caption}
%\usepackage[titletoc]{appendix}


% vältä marginaalin ylitystä
%\setlength{\emergencystretch}{3em}


\begin{document}
%\raggedright % Tasaamattomat oikeat kappaleet. Voi pitää tasaamattomina!

\selectlanguage{english}

\frontmatter
\pagestyle{plain}


%%% COVER PAGE

\author{Eric Malmi, Hannu Hartikainen}
\title{Ensemble Method for Spam Classification}

\maketitle



\clearpage


%%% ABSTRACT

%\input{abstract}
\clearpage


%%% TABLE OF CONTENTS 

\tableofcontents
\label{pages-prelude}
\clearpage


% the main content starts
\mainmatter

% Muutetaan ala- ja ylätunnisteet
\pagestyle{headings}


% CHAPTERS
%\input{introduction}
%\input{...}

\clearpage


%%% BIBLIOGRAPHY

% näytetään kaikki lähteet
\nocite{*}

\addcontentsline{toc}{chapter}{\bibname}
%\addcontentsline{toc}{chapter} {References}

% Tässä voit valita, mitä viittauskäytäntöä käytät.
% Nyt kun kommentoitu pois, niin Harvard-tyyppinen "nimi-vuosi"
\bibliographystyle{plainnat}
%\bibliographystyle{plain}     % ... tutkimuksessa [1].
%\bibliographystyle{alpha}     % ... tutkimuksessa [Meik09]
%\bibliographystyle{apalike}   % ... tutkimuksessa 

\bibliography{sources}


%%% APPENDICES

%\begin{appendices}
%\input{...}
%\end{appendices}

\end{document}
