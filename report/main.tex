\documentclass[12pt,a4paper,finnish,oneside]{article}
%\documentclass[12pt,a4paper,finnish,oneside,draft]{book} % luonnos, nopeampi
%\usepackage{tkk_t}
\usepackage[utf8]{inputenc}
\usepackage[T1]{fontenc}
\usepackage{alltt}
\usepackage[english,swedish,finnish]{babel}
\usepackage{amsmath}  % matematiikkaa..
\usepackage[pdfborder={0 0 0}]{hyperref}
\usepackage{color}
\usepackage{listings}
\usepackage{rotating}
\usepackage[format=hang,justification=raggedright]{caption}
%\usepackage[titletoc]{appendix}


% vältä marginaalin ylitystä
%\setlength{\emergencystretch}{3em}


\begin{document}
%\raggedright % Tasaamattomat oikeat kappaleet. Voi pitää tasaamattomina!

\selectlanguage{english}

%\frontmatter
\pagestyle{plain}


%%% COVER PAGE

\author{Eric Malmi, Hannu Hartikainen}
\title{Ensemble Method for Spam Classification}

\maketitle



\clearpage


%%% ABSTRACT

% TODO: proper handling of abstract
\section*{Abstract}

Classification algorithms have each their own characteristics, which
have both strengths and weaknesses. By combining different classifiers,
inherent weaknesses of a single classifier can be overcome. We propose
an ensemble method for classifying binary data, combining predictions of
three different classifiers. The Support Vector Machine, Bernoulli
Mixture and Random Forest classifiers are used, each outputting
prediction values in the continuous range $[0,1]$. These predictions are
combined by the arithmetic mean.

A data set of 10000 e-mail messages containing classes \emph{spam} and
\emph{ham} was used for testing the ensemble method. The validation
accuracy of the ensemble classifier was significantly better than the
best individual classifier, while the test accuracy was only very
slightly better. The test accuracy achieved by the ensemble classifier
was $98.0 \%$.

\clearpage


%%% TABLE OF CONTENTS 

\tableofcontents
\label{pages-prelude}
\clearpage


% the main content starts
%\mainmatter

% Muutetaan ala- ja ylätunnisteet
\pagestyle{headings}


%%% CHAPTERS
\section{Introduction}

The enormous growth of the Internet in the recent years has given ground
to many new kinds of businesses. There have been great benefits for
users from new, useful and exciting services. But in the meantime,
advertising has also found new forms, and not all of them are good.

\emph{Spam}, or unsolicited e-mail, has been a major headache for many
an e-mail user in the past decade. Sending e-mail costs nearly nothing,
so massive amounts of spam are sent around. Finding the useful e-mail
messages (\emph{ham}) has quickly become a laborious and daunting task
for a human. Luckily, \emph{spam filter} software has been developed and
nowadays works very well.

Filtering spam messages from e-mail is an interesting machine learning
problem. Our term project for \emph{T--61.3050 Machine Learning: Basic
Principles} is to adapt machine learning algorithms for classifying
e-mail messages as \emph{spam} or \emph{ham}. The data is not actual
e-mail messages, but binary data describing features found in the
messages. This data was originally generated by SpamAssassin.

We apply an ensemble method for the classification task. The ensemble
consists of three different types of classifiers complementing each
other, namely a \emph{support vector machine}, a \emph{Bernoulli
mixture} and a \emph{Random Forest classifier}. These methods are
described in Section \ref{sec:methods}. In Section
\ref{sec:experiments}, we proceed to explain the experimental setting we
had and the results are given in Section \ref{sec:results}. Finally, we
draw conclusions based on the results and discuss some possible
improvements in Section \ref{sec:discussion}.

\section{Methods}

Choosing an algorithm for classifying data is difficult, because
different algorithms have different characteristics. There's no single
algorithm that's always best---each algorithm has its own strengths and
weaknesses. With this realization, we decided to use an ensemble method,
combining different algorithms.

We decided to choose three algorithms, so as to gain some redundancy
while keeping the method relatively simple. For any type of failure
\emph{intrinsic} to an algorithm, the two other algorithms are not prone
to make the same mistake. Thus most intrinsic mis-classifications can be
avoided; while one algorithm claims the wrong class, the two others get
it right.

The three algorithms we use all output \emph{probabilities}. Even in the
case that two of the algorithms are uncertain but leaning towards the
wrong classification, but one is correct and very certain, we can get
the correct result.

For combining the predictions of different algorithms, we used simply
the arithmetic mean. In our experiments, this proved to be efficient and
no other method showed significantly better results.

\subsection{Support Vector Machine}

The Support Vector Machine classifier works by finding a hyperplane that
separates two data samples with the largest possible margin. The method
becomes efficient as data is projected to a higher-dimension space.
According to Cover's Theorem, this makes finding a separating hyperplane
more probable. The actual mapping between the two spaces need not be
found--it suffices to define the dot product, i.e.~the kernel function
$\phi(u,v)$. \cite{cortes1995support}

Commonly, the kernel function is chosen to be linear $\phi(u,v) = u' v$,
polynomial $\phi(u,v) = (\gamma u' v + C)^n$, radial basis function
$\phi(u,v) = e^{-\gamma |u-v|^2}$ or sigmoid
$\phi(u,v) = tanh(\gamma u' v + C)$.

Weka \cite{weka} was used for applying the Support Vector Machine
classifier to the data. The specific implementation we used was LibSVM
\cite{libsvm}, using the WLSVM interface \cite{wlsvm}.

\subsection{Bernoulli Mixture}

If we know the likelihood functions $p(\mathbf{x}|\boldsymbol\theta_i)$
and prior probabilities $P(C_i)$ of classes $i$ we want to distinguish,
the Bayes' rule gives us an optimal classifier. The class of a new
instance $\mathbf{x}$ is given by choosing the class with the highest
posterior probability
\begin{equation}
 \textmd{arg}\max_i P(C_i|\mathbf{x}) \propto p(\mathbf{x}|\boldsymbol\theta_i)P(C_i).
\end{equation}

In reality, however, we do not have the correct likelihood functions but
we need to estimate them from the data. This is what we do in this work
as well, we estimate likelihood functions for both ham and spam
instances and then use these models to classify new instances.

A natural way to model binary data and estimate its likelihood function
is to use a Bernoulli distribution, given by
\begin{equation}
  p(\mathbf{x}|\boldsymbol\mu) = \prod_{i=1}^D \mu_i^{x_i}(1-\mu_i)^{(1-x_i)}.
\end{equation}

Parameters $\mu_i$ are the means of different variables in data
$\mathbf{x}$ and thus the farther from the means we are the lower
likelihood we get.

A drawback of the Bernoulli distribution is, however, that it assumes
that the variables are independent. In case of separating between spam
and ham, this assumption leads into in a Naive Bayes classifier. The
reason why this assumption is nevertheless commonly approved is that it
is infeasible to model all the covariances between variables as this
would require us to estimate a total of $\mathcal{O}(n^2)$ parameters.

To circumvent these problems, we use a
\emph{mixture of multivariate Bernoulli distributions} or shortly a
\emph{Bernoulli mixture} \cite{lazarsfeld1968latent}. A Bernoulli
mixture takes a linear combination of single Bernoullis, called the
mixture components, with parameters $\boldsymbol\mu_i$. The resulting
distribution is given by
\begin{equation}
 p(\mathbf{x}|\boldsymbol\mu,\boldsymbol\pi) = \sum_{k=1}^K \pi_kp(\mathbf{x}|\boldsymbol\mu_k),
\end{equation}

where $\pi_k$ are the weights that sum up to one. The covariance of this
distribution is not anymore a diagonal matrix meaning that it captures
some of the correlations in the data \cite{bishop2006pattern}.

To learn the model, we use the BernoulliMix software package
\cite{bmix}. BernoulliMix uses the EM algorithm to learn the model
parameters.

\subsection{Random Forest}

The Random Forest classifier, described by Ho \cite{ho1995random}, works
by creating a bunch of decision trees randomly. Each single tree is
created in a randomly selected subspace of the feature space. Trees in
different subspaces complement each other's classifications.

The single trees are Oblique Decision Trees. For each node of a tree, a
hyperplane is selected to divide the sample into two further
subspaces--corresponding to two branches of the tree. This is repeated
until each branch ends with a subspace containing only a single class.

Predictions of the different trees for a test point are combined with a
discriminant function, which in \cite{ho1995random} is basically done by
just averaging the posterior probabilities. The Random Forest classifier
attains an improved generalization accuracy compared to other decision
tree-based classifiers. It still retains the excellent accuracy on
training data inherent to decision trees. \cite{ho1995random}

The built-in Random Forest implementation in Weka \cite{weka} was used.
The implementation is created considering some further research on
generating Random forests \cite{breiman2001random}.

\section{Experiments} \label{sec:experiments}

Having chosen the classifiers, we still had some possibilities for
improving accuracy. Each of the classifiers had parameters that could be
optimized. Also, the method of combining predictions could be freely
chosen.

\subsection{Parameter optimization}

For testing accuracy we had the 1000 known data points available. To
make the best use of the data, we used cross-validation for training and
testing the classifiers. Cross-validation means making different
partitions of the data, each time choosing a large training set and a
small test set. This way, all data points get a prediction in one of the
partitions. These different test results are then averaged to get
results representing the whole known data set.

As a compromise between performance and reliability, we chose to use
10-fold cross-validation. This means that the data is split into 10
equally sized subsamples. The classifier is then run 10 times, each time
choosing 9 subsamples as a 900-item training set and the rest as a
100-item test set.

The cross-validation method we used was also \emph{stratified}, meaning
that each of the subsamples was randomly chosen so that it includes the
same distribution of spam and ham as the original sample.

\subsubsection{Support Vector Machine}

The LibSVM implementation of the Support Vector Machine classifier can
be tuned with many different parameters. The most important parameter is
the choice of kernel function. We briefly tried out each of the four
functions that LibSVM includes (linear, polynomial, radial basis
function and sigmoid). The default radial basis function clearly
outperformed the others in terms of accuracy. The radial basis function
is of the form

\begin{equation}
\exp(-\gamma |u-v|^2)
\end{equation}

where $u$ and $v$ are the data points and $\gamma$ is a freely
selectable parameter. The value of $\gamma$ was optimized using 10-fold
cross-validation and maximizing the average accuracy. The output of
LibSVM was changed to probability estimates instead of hard binary
classifications.

\subsubsection{Bernoulli mixture}

The optimizable parameters in a Bernoulli mixture are the number of
components for the ham model and for the spam model. We varied the
number of these components separately from 1 to 20 and used 10-fold
cross-validation to measure the accuracy of each parameter combination.
Each partitioning was rerun 5 times to get a more reliable estimate of
the accuracy since the EM algorithm gets sometimes stuck in a local
optimum. The accuracy was calculated as the average over all these runs.
It might have been better to select only the best of the 5 runs.

\subsubsection{Random Forest}

The Random Forest classifier allowed changing tree generation
parameters, the amount of trees in the forest and the random seed. As
the trees are independent of each other, we reasoned that the amount of
trees affects the variance of accuracy. We chose to minimize variance by
adding more trees. Otherwise, we chose to use the default parameters as
provided by Weka 3.6.5.

\subsection{Ensemble validation}

For validating the ensemble method, we split the known data in half as a
training set and a test set. The sets were chosen so that the
distribution of spam and ham was retained. The reason for not using
10-fold cross-validation for validating the ensemble method as well as
the single classifiers was related to time constraints. This should be
changed in future work.

The baseline method for combining predictions was chosen to be simply
the arithmetic mean. This seemed the simplest and most intuitive, so
other methods were compared to it. Besides mean, we also tested the
median of the predictions, and the mode of rounded binary predictions.
Both yielded worse accuracy in the test set, so they were disregarded.

We also thought of using a weighted mean. The errors each classifier
made in the test set were manually reviewed. However, we found no
significantly better way to adjust the weights than to just use the same
weight for all three classifiers.

\section{Results}

\subsection{Parameter Optimization}

For SVM, we optimized parameter $\gamma$ and found the optimum to be
$\gamma = 0.3$. Changing other parameters from their defaults did not
yield notable improvements. Otherwise, we used the default parameter
values that Weka 3.6.5 provides for LibSVM.

The accuracies of different parameter combinations for the Bernoulli
mixture are shown in Figure (lisää kuva results/bmix\_opt.pdf). The best
accuracy (98.0 \%) was obtained with 12 ham components and 17 spam
components.

With the Random Forest classifier, we chose to use the default
parameters as changing these did not seem to affect the classification
accuracy significantly.

\subsection{Classification}

The average cross-validation accuracy of each classifier for the known
data is shown in Table \ref{results1k}.

\begin{table}[H]
\caption{Average accuracy on known data (10-fold cross-validation)}
\label{results1k}
%\begin{center}
\begin{tabular}{l|c}
Classifier & Accuracy \\ \hline
SVM & 98.4\% \\
Bernoulli mixture & 98.0\% \\
Random forest & 97.1\% \\
\end{tabular}
%\end{center}
\end{table}

The ensemble method gave us $99.2 \%$ accuracy with a simple half-half
split of the known data that was used also for the parameter
optimization. Test accuracies and error counts for unknown 9000 data
samples are given in Table \ref{results9k}.

\begin{table}
\caption{Accuracy and error count on unknown data}
\label{results9k}
%\begin{center}
\begin{tabular}{l|c|c}
Classifier & Accuracy & Error count \\ \hline
SVM & 97.2\% & 251 \\
Bernoulli mixture & 98.0\% & 181 \\
Random forest & 96.3\% & 329 \\ \hline
Ensemble method & 98.0\% & 179 \\
\end{tabular}
%\end{center}
\end{table}

\section{Discussion} \label{sec:discussion}

In the Bernoulli mixture, we weighted the classes' posterior
probabilities with the classes' prior probabilities. This seemed to
improve the Bernoulli mixture as it superseded the other two methods in
test accuracies. Similar weighting could have been done with the other
two methods. However, this would not have been so well justified as the
other methods are not probabilistic by nature and it is arguable whether
we can interpret their outputs as probabilities. Therefore, it is not
meaningful to try draw any probabilistic interpretations for taking the
mean of the classifiers' outputs. Instead, this operation should be
regarded as a simple heuristic trick that proved to be useful.

The 448 dimensional data we classified contained probably lots of
insignificant or redundant information that did not contribute to the
classification. Therefore, some feature extraction methods could have
been tested to make the classification task easier for the different
classifiers. For example, the EM algorithm used with the Bernoulli
mixture is time consuming and may find only local optima. The parameter
optimization, whose results are shown in Figure \ref{fig:bmix}, took
about one week to complete with a standard desktop and the accuracies
over different folds varied. More parameter combinations and repetitions
could have been calculated after applying, e.g., binary PCA to the data.
An alternative approach would have been to use feature selection where
we try find only the variables that contribute to the classification
task.

Another ways to improve the classification could have been to try out a
more broad set of base learners and different ways of combining these.
For example, boosting or bagging could have been tested.

\subsection{Workload}

The project was moderately challenging. Granted, it could have been less difficult had we chosen a simpler method. Each of us used approximately 25 hours\footnote{about 1 ECTS credit} for the project. The total is thus 50--55 hours.

\section{Conclusions}

We applied an ensemble method for classifying multivariate binary data
representing ham and spam messages. Three state-of-the-art classifiers,
namely a support vector machine, a Bernoulli mixture and a Random Forest
classifier were used, choosing the model parameters with 10-fold
cross-validation. High cross-validation and test accuracies ranging from
$96.3 \%$ to $98.4 \%$ were obtained with each of the classifiers.

The classifiers gave continuous output values from 0 to 1 and the
ensemble was selected as the mean of these values. The ensemble
validation accuracy ($99.2 \%$) clearly superseded that of each
individual classifier. The ensemble test accuracy was equal to the test
accuracy of Bernoulli mixture which had the best performance on the test
set. Consequently, we succeeded in our mission to select versatile
classifiers that complement each other.


\clearpage


%%% BIBLIOGRAPHY

% näytetään kaikki lähteet
\nocite{*}

\addcontentsline{toc}{chapter}{\bibname}
%\addcontentsline{toc}{chapter} {References}

% Tässä voit valita, mitä viittauskäytäntöä käytät.
% Nyt kun kommentoitu pois, niin Harvard-tyyppinen "nimi-vuosi"
\bibliographystyle{plainnat}
%\bibliographystyle{plain}     % ... tutkimuksessa [1].
%\bibliographystyle{alpha}     % ... tutkimuksessa [Meik09]
%\bibliographystyle{apalike}   % ... tutkimuksessa 

\bibliography{sources}


%%% APPENDICES

%\begin{appendices}
%\input{...}
%\end{appendices}

\end{document}
