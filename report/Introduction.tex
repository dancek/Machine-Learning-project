\section{Introduction}

The enormous growth of the Internet in the recent years has given ground
to many new kinds of businesses. There have been great benefits for
users from new, useful and exciting services. But in the meantime,
advertising has also found new forms, and not all of them are good.

\emph{Spam}, or unsolicited e-mail, has been a major headache for many
an e-mail user in the past decade. Sending e-mail costs nearly nothing,
so massive amounts of spam are sent around. Finding the useful e-mail
messages (\emph{ham}) has quickly become a laborious and daunting task
for a human. Luckily, \emph{spam filter} software has been developed and
nowadays works very well.

Filtering spam messages from e-mail is an interesting machine learning
problem. Our term project for \emph{T--61.3050 Machine Learning: Basic
Principles} is to adapt machine learning algorithms for classifying
e-mail messages as \emph{spam} or \emph{ham}. The data is not actual
e-mail messages, but binary data describing features found in the
messages. This data was originally generated by SpamAssassin.

We apply an ensemble method for the classification task. The ensemble
consists of three different types of classifiers complementing each
other, namely a \emph{support vector machine}, a \emph{Bernoulli
mixture} and a \emph{Random Forest classifier}. These methods are
described in Section \ref{sec:methods}. In Section
\ref{sec:experiments}, we proceed to explain the experimental setting we
had and the results are given in Section \ref{sec:results}. Finally, we
draw conclusions based on the results and discuss some possible
improvements in Section \ref{sec:discussion}.
